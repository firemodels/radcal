% !TEX root = RadCal_User_Guide.tex

\typeout{new file: Introduction_Chapter.tex}

\chapter{Introduction}

Thermal radiation fire plays a preponderant role in most fires as it is responsible for fire spread and fire growth due to thermal feedback from the hot upper layer in a compartment fire or the hot plume in an open pool fire, or controls the mass loss rate in large scale pool fire.
In all the aforementioned cases, it is crucial to accurately determinate the radiative heat transfer. The use of the Wien's displacement law quantifies the wavelength of the maximum radiative energy emitted by a blackbody. This law is expressed
by
\begin{equation}\label{eq:Wien}
 \rm \lambda_m T = 2897 \, \mu m.K
\end{equation}
with $\rm \lambda_m$ being the wavelength in units of m, corresponding to the peak intensity of blackbody emittance; T represents the temperature in units of Kelvin of the blackbody. Equation \ref{eq:Wien} says that in typical fire configurations, where the temperature ranges from 300~K to about 2500~K, the peak emittance wavelength varies from 10 $\rm \mu m$ to 1.1 $\rm \mu m$: almost all the radiative exchanges happen in the near to mid infrared range. Most of the components involved in fire are present in gas phase. The infrared spectrum of the gases is very discontinuous, with few narrow spectral areas that participate to radiative exchange. The rest of the spectrum is practically transparent. The participation propensity varies with the amount of the species present and the local temperature.

These intrinsics aspects of the radiative exchange between medium with important gradients of temperature, species amount, and non-homogeneity in species render this problem quite complex and some level of sophistication is needed in its treatment. The radiative properties of a gas species are dictated by quantum mechanics and are related to the composition and the structure of the component considered. The discrete nature of the energy levels (mostly vibrational and rotational modes) for a given molecule generate its infrared ``fingerprint''. While a thorough consideration of all the energy levels would give an exact assessment of the radiative exchange -- this is often referred to as line-by-line calculation in the literature -- this operating mode is still too computationally expensive for engineering applications and is only used for simple configurations. Moreover, the lack of data for both elevated temperatures and for most hydrocarbon species further restricts its applicability to fire scenarios. The development of narrow-band models constitutes a compromise between accuracy and efficiency. Narrow-band models divide the spectrum of interest into small spectral segments of uniform spectral properties and use statical representations of the energy levels over these segments. Narrow band models are computationally fast and accurate. In particular, they do not require detailed knowledge of all the active energy levels in a given molecule. They are easier to implement and use than Line-by-Line techniques.

RadCal was previously developed by Grosshandler \cite{Grosshandler1993} to predict radiative heat transfer from gases at elevated temperature using narrow-band models. RadCal is a computer program, originally written in FORTRAN 77, that computes the directional spectral intensity from a non-isothermal, non-uniform mixture of gases and soot, by spectrally solving the radiative transfer equation. In addition, RadCal returns the Planck mean absorption coefficient, an effective absorption coefficient, and other integrated quantities. Details about the different models used and the quantities printed by RadCal are provided in Chapter \ref{chap:SNB}.

The first version of RadCal was developed to predict the enhancement in radiation caused by the addition of pulverized coal to a 60 kW methanol-fired furnace~\cite{Grosshandler1976}. This first version considered the contributions of CO, $\rm CO_2$, $\rm H_2O$, and soot. Validation of first version of RadCal was documented in 1979 and can be found in Ref.~\cite{Grosshandler1979}. Predictions from RadCal for $\rm CO_2$, CO, and $\rm H_2O$ in individual or in mixtures were compared against published data. Good agreement was found except for some data. As Grosshandler states in Ref.~\cite{Grosshandler1993}:
\textit{``...The spectrum between 1.25 and 12.5 $\mu$m was satisfactorily reproduced, although some of the data at particular wavelengths differed from the prediction by as much as 17\%. Considerable disagreement occurred between the integrated emittance of $\rm CO_2$ as predicted from RADCAL and that computed from the charts of Hottel \cite{Hottel1954}. No one source for this disagreement was identified, but it was thought to be a combination of the difficulty in obtaining high accuracy spectral measurements under the full range of conditions investigated, the uncertainty associated with extrapolating total transmittance results beyond the measured temperature and pressure-pathlengths, and the approximations associated with the narrow-band models.''}

Methane was added to RadCal in 1985 \cite{Grosshandler1985}, along with an extension to 200 $\rm \mu m$ of the considered spectrum. The added methane data originates mostly from experiments performed by Brosmer \textit{et al.} \cite{Brosmer1985} and Lee \textit{et al.} \cite{Lee1964}. At this time, the code structure was updated and a new input file was created.

This report, a second edition of NIST Special Publication 1402, presents the latest enhancements brought to RadCal and aims to provide an exhaustive list of the mathematical and physical models used in RadCal, which was missing from the first edition. Chapter~\ref{chap:SNB} presents the fundamental mathematical models used in radiative heat transfer and presents the different narrow-band models used in RadCal. Chapter~\ref{chap:old_species} recalls the characteristics of the species that were present in the 1993 version of RadCal: $\rm H_2O$, $\rm CO_2$, CO, $\rm CH_4$, and soot. Additional hydrocarbons have been implemented into RadCal and the code has been rewritten, for its most part, into Fortran 2008. FTIR transmission measurements at the National Institute of Standards and Technology (NIST) were undertaken to provide highly resolved spectral absorption coefficients in the mid-IR and NIR as a function of temperature for many fuel species~\cite{Wakatsuki2005a,Wakatsuki2008,Yilmaz2008}. These measurements were performed over a range of temperatures from 300~K up to 1000~K for several fuel species, including paraffins (methane, ethane, propane, and n-heptane), olefins (ethylene and propylene), and other fuel-related species (methanol, toluene, and methyl-methacrylate). The uniform set of conditions and spectral resolution of these measurements have provided a set of data for developing calculation methodologies for absorption coefficients of these species in flame environments. New species and their associated narrow-band parameters are described in Chapter~\ref{chap:new_species}. The syntax of the input file was modified to make good use of native Fortran namelist that offers a more flexible way to input data. Chapter~\ref{chap::using_RADCAL} describes the new input file syntax, and how to compile and use RadCal. The code was modified following a modular approach and was translated into Fortran 2008 to benefit from recent advances in Fortran standard. The list of RadCal functions and subroutines is detailed in Chapter~\ref{chap:Code}.  The code has also been modified to account for any type of fuel mixture. Verification of the new species spectral data has been performed and results from these tests are reported in Chapter~\ref{sec:verification}. Finally, Chapter~\ref{chap::validation_tests} compares the new version of RadCal with the 1993 one for the various validation tests presented in the first edition of this report. This chapter also presents predicted quantities from an experimentally characterized small methanol pool fire.



Include here New RadCal features in a list:

\begin{itemize}
 \item New species spectral data between 700 and 4000~$\rm cm^{-1}$ and for temperature ranging from about 300~K to about 1000~K.
 \item New species data for: Ethylene, Ethane, Propylene, Propane, n-Heptane, Toluene, Methanol, Methane, MMA
 \item Code rewritten in Fortran 2008
 \item Code made modular
 \item More convenient input file
 \item Possibility to calculate the spectrum for any mixture of gas
 \item Updated and enhanced user guide
\end{itemize}
